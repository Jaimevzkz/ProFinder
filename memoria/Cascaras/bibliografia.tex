%---------------------------------------------------------------------
%
%                      configBibliografia.tex
%
%---------------------------------------------------------------------
%
% bibliografia.tex
% Copyright 2009 Marco Antonio Gomez-Martin, Pedro Pablo Gomez-Martin
%
% This file belongs to the TeXiS manual, a LaTeX template for writting
% Thesis and other documents. The complete last TeXiS package can
% be obtained from http://gaia.fdi.ucm.es/projects/texis/
%
% Although the TeXiS template itself is distributed under the 
% conditions of the LaTeX Project Public License
% (http://www.latex-project.org/lppl.txt), the manual content
% uses the CC-BY-SA license that stays that you are free:
%
%    - to share & to copy, distribute and transmit the work
%    - to remix and to adapt the work
%
% under the following conditions:
%
%    - Attribution: you must attribute the work in the manner
%      specified by the author or licensor (but not in any way that
%      suggests that they endorse you or your use of the work).
%    - Share Alike: if you alter, transform, or build upon this
%      work, you may distribute the resulting work only under the
%      same, similar or a compatible license.
%
% The complete license is available in
% http://creativecommons.org/licenses/by-sa/3.0/legalcode
%
%---------------------------------------------------------------------
%
% Fichero  que  configura  los  parámetros  de  la  generación  de  la
% bibliografía.  Existen dos  parámetros configurables:  los ficheros
% .bib que se utilizan y la frase célebre que aparece justo antes de la
% primera referencia.
%
%---------------------------------------------------------------------


%%%%%%%%%%%%%%%%%%%%%%%%%%%%%%%%%%%%%%%%%%%%%%%%%%%%%%%%%%%%%%%%%%%%%%
% Definición de los ficheros .bib utilizados:
% \setBibFiles{<lista ficheros sin extension, separados por comas>}
% Nota:
% Es IMPORTANTE que los ficheros estén en la misma línea que
% el comando \setBibFiles. Si se desea utilizar varias líneas,
% terminarlas con una apertura de comentario.
%%%%%%%%%%%%%%%%%%%%%%%%%%%%%%%%%%%%%%%%%%%%%%%%%%%%%%%%%%%%%%%%%%%%%%
\setBibFiles{%
biblio%
}

%%%%%%%%%%%%%%%%%%%%%%%%%%%%%%%%%%%%%%%%%%%%%%%%%%%%%%%%%%%%%%%%%%%%%%
% Definición de la frase célebre para el capítulo de la
% bibliografía. Dentro normalmente se querrá hacer uso del entorno
% \begin{FraseCelebre}, que contendrá a su vez otros dos entornos,
% un \begin{Frase} y un \begin{Fuente}.
%
% Nota:
% Si no se quiere cita, se puede eliminar su definición (en la
% macro setCitaBibliografia{} ).
%%%%%%%%%%%%%%%%%%%%%%%%%%%%%%%%%%%%%%%%%%%%%%%%%%%%%%%%%%%%%%%%%%%%%%
\setCitaBibliografia{
\begin{FraseCelebre}
\begin{Frase}
  Nobody actually creates perfect code the first time around, except me. But there's only one of me.
\end{Frase}
\begin{Fuente}
  Linus Torwalds
\end{Fuente}
\end{FraseCelebre}
}

\newpage
URLs referenciadas
\begin{itemize}
  \item \url{https://developer.android.com/studio/intro}
  \item \url{https://www.jetbrains.com/idea}
  \item \url{https://www.java.com}
  \item \url{https://www.scala-lang.org}
  \item \url{https://obsidian.md}
  \item \url{https://www.figma.com}
  \item \url{https://m3.material.io}
  \item \url{https://m3.material.io/styles/color/system/overview}
  \item \url{https://www.drawio.com/}
  \item \url{https://git-scm.com/}
  \item \url{https://github.com/}
  \item \url{https://github.com/jesseduffield/lazygit}
  \item \url{https://kotlinlang.org/}
  \item \url{https://www.plainconcepts.com/es/kotlin-android/}
  \item \url{https://developer.android.com/develop/ui/compose}
  \item \url{https://kotlinlang.org/docs/lambdas.html}
  \item \url{https://dagger.dev/hilt/}
  \item \url{https://www.fsf.org/}
  \item \url{https://www.gnu.org/}
  \item \url{https://www.mozilla.org/es-ES/firefox/}
  \item \url{https://neovim.io/}
  \item \url{https://www.linux.org/pages/}
  \item \url{https://developer.android.com/topic/libraries/architecture/viewmodel}
  \item \url{https://builtin.com/software-engineering-perspectives/mvvm-architecture}
  \item \url{https://developer.android.com/guide/components/intents-filters}
  \item \url{https://www.youtube.com/watch?v=MiLN2vs2Oe0}
  \item \url{https://kotlinlang.org/docs/control-flow.html#when-expression}
  \item \url{}
  \item \url{}
  \item \url{}
  \item \url{}
  \item \url{}
  \item \url{}
  \item \url{}
  \item \url{}
  \item \url{}
\end{itemize}

%%
%% Creamos la bibliografia
%%
\makeBib

% Variable local para emacs, para  que encuentre el fichero maestro de
% compilación y funcionen mejor algunas teclas rápidas de AucTeX

%%%
%%% Local Variables:
%%% mode: latex
%%% TeX-master: "../Tesis.tex"
%%% End:
