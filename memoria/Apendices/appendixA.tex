\chapter{Especificación de requisitos}
\label{Appendix:srs}

En este apendice se especifican los actores y casos de uso de la aplicación Profinder. Se trata de una especificación inicial por lo que tanto actores como casos de uso han sido cambiados, ampliados o recortados a lo largo del proceso de desarrollo.

\section{Actores}
A continuación se describen los distintos actores de la aplicación:
\begin{itemize}
    \item \textbf{Usuario}: Un usuario es cualquier persona que se dé de alta como tal, debe haberse creado una cuenta e iniciado sesión para ser identificado, estos son los que contratan los servicios que los profesionales ofrecen, entre otras funciones los usuarios pueden ver el mapa de profesionales en tiempo real y filtrar los datos que le aparecen según la categoría de trabajo seleccionada.
    \item \textbf{Profesional}: Un profesional es cualquier persona que se dé de alta como tal, su proceso de registro es ligeramente diferente al de un usuario normal ya que deben especificar la categoría en la que son profesionales, así como otros detalles que sean importantes para su profesión.
    \item \textbf{Administrador}: El administrador de la aplicación es el que se encarga de la organización y el correcto funcionamiento de la misma, podrá añadir o eliminar categorías, bloquear usuarios y/o profesionales, así como responder a mensajes dirigidos a soporte, los administradores deben ser asignados una vez creada la cuenta desde fuera de la aplicación.
\end{itemize}
\section{Casos de Uso}
Los casos de uso se han clasificado en cuatro categorías según el actor principal del mismo.
\newpage
\subsection{Casos de uso generales}
\begin{table}[!h]
	\begin{tabularx}{\textwidth}{|c|X|}
	\rowcolor[HTML]{00D2CB} 
	\hline          
	\textbf{Requisito} & \textbf{Registro/baja} \\
	\hline
	Identificador & 1.1 \\
	\hline
	Prioridad & Alta \\
	\hline
	Precondición & En caso de baja: tener una cuenta creada. \\
	\hline
	Descripción & Los actores de la aplicación tendrán que crear una cuenta para poder interactuar con la misma, asimismo tendrán la capacidad de dar de baja su cuenta cuando lo deseen. \\
	\hline
	Entrada & Nombre de usuario, correo electrónico, contraseña. \\
	\hline
	Salida & Actor registrado/dado de baja \\
	\hline
	Secuencia normal & \begin{tabular}{@{}p{1cm}|p{9.5cm}@{}}
		Paso & Acción \\
		\hline  
		1 & El actor abre la aplicación y se le muestra la pantalla para iniciar sesión con un botón específico para registrarse. \\
		\hline  
		2 & se selecciona la opción ‘Registrarse’. \\
		\hline  
		3 & El actor rellena el formulario con los datos de entrada y pulsa ‘Enviar’. \\
		\hline  
		4 & El sistema procesa el formulario y crea la cuenta. \\
		\end{tabular} \\
	\hline
	Postcondición & Se ha creado la cuenta. \\
	\hline
	Excepciones & \begin{tabular}{@{}p{1cm}|p{9.5cm}@{}}
		Paso & Acción \\
		\hline  
		4 & Los datos introducidos son incorrectos o no cumplen los requisitos. No se crea la cuenta y se avisa al usuario. \\
		\end{tabular}  \\
	\hline
	Comentarios & Este caso de uso permite registrar o dar de baja de la aplicación a actores de la misma. \\
	\hline
	Actores & Usuario, Profesional \\
	\hline            
	\end{tabularx}
	\caption{Registro/baja}
	\label{tab:cu_1}  
\end{table}
%---------------------------------------------------------------
\newpage
\begin{table}[!h]
	\begin{tabularx}{\textwidth}{|c|X|}
	\rowcolor[HTML]{00D2CB} 
	\hline          
	\textbf{Requisito} & \textbf{Modificar datos} \\
	\hline
	Identificador & 1.2 \\
	\hline
	Prioridad & Media \\
	\hline
	Precondición & Haber iniciado sesión. \\
	\hline
	Descripción & Una vez creada una cuenta, tanto profesionales como usuarios podrán modificar sus datos de perfil. \\
	\hline
	Entrada & Datos a cambiar. \\
	\hline
	Salida & Nuevos datos. \\
	\hline
	Secuencia normal & \begin{tabular}{@{}p{1cm}|p{9.5cm}@{}}
		Paso & Acción \\
		\hline  
		1 & El actor se dirigirá al apartado de ‘Mi perfil’ y seleccionará la opción ‘Modificar perfil’. \\
		\hline  
		2 & Dentro de esta pantalla se modificarán todos los datos deseados. \\
		\end{tabular} \\
	\hline
	Postcondición & Se han modificado los datos de perfil deseados. \\
	\hline
	Excepciones & \begin{tabular}{@{}p{1cm}|p{9.5cm}@{}}
		Paso & Acción \\
		\hline  
		2 & Los nuevos datos no son válidos. No se guarda la modificación. \\
		\end{tabular}  \\
	\hline
	Comentarios & Este caso de uso permite modificar datos de su perfil a los actores. \\
	\hline
	Actores & Usuario, Profesional \\
	\hline            
	\end{tabularx}
	\caption{Modificar datos}
	\label{tab:cu_2}  
\end{table}
%---------------------------------------------------------------
\begin{table}[tbph]
	\begin{tabularx}{\textwidth}{|c|X|}
	\rowcolor[HTML]{00D2CB} 
	\hline          
	\textbf{Requisito} & \textbf{Login/Logout} \\
	\hline
	Identificador & 1.3 \\
	\hline
	Prioridad & Alta \\
	\hline
	Precondición & Haber creado una cuenta previamente, en caso de logout tener sesión iniciada. \\
	\hline
	Descripción & Para poder interactuar con la aplicación, tanto usuarios como profesionales tendrán que iniciar sesión en la aplicación. Los usuarios que ya hayan iniciado sesión y quieran cerrarla tendrán la opción de hacerlo. \\
	\hline
	Entrada & En caso de login: correo electrónico/nombre de usuario, contraseña. \\
	\hline
	Salida & N.A. \\
	\hline
	Secuencia normal de login & \begin{tabular}{@{}p{1.5cm}|p{7.2cm}@{}}
		Paso & Acción \\
		\hline  
		1 & El actor abre la aplicación y se le muestra la pantalla de  ‘Iniciar sesión’. \\
		\hline  
		2 & Se muestra una pantalla con el formulario de inicio de sesión. El usuario introduce los datos de entrada y pulsa el botón ‘Iniciar sesión’. \\
		\end{tabular} \\
	\hline
	Secuencia normal de logout & \begin{tabular}{@{}p{1.5cm}|p{7.2cm}@{}}
		Paso & Acción \\
		\hline  
		1 & El actor se dirige a la sección ‘Mi perfil’ donde se le mostrarán múltiples opciones de gestión de su cuenta. \\
		\hline  
		2 & Dentro de estas opciones se encontrará la opción ‘Cerrar sesión’. El actor pulsará el botón. \\
		\end{tabular} \\
	\hline
	Postcondición & Se ha iniciado sesión/se ha cerrado sesión. \\
	\hline
	Excepciones & \begin{tabular}{@{}p{1.5cm}|p{7.2cm}@{}}
		Paso & Acción \\
		\hline  
		2 (login) & Los campos rellenados no concuerdan con ningún actor del sistema. No se completa el login. \\
		\end{tabular}  \\
	\hline
	Comentarios & Este caso de uso permite iniciar sesión en la aplicación, así como cerrar la misma. \\
	\hline
	Actores & Usuario, Profesional, Administrador \\
	\hline            
	\end{tabularx}
	\caption{Login/Logout}
	\label{tab:cu_3}  
\end{table}
%---------------------------------------------------------------
\newpage
\begin{table}[!h]
	\begin{tabularx}{\textwidth}{|c|X|}
	\rowcolor[HTML]{00D2CB} 
	\hline          
	\textbf{Requisito} & \textbf{Configurar app} \\
	\hline
	Identificador & 1.4 \\
	\hline
	Prioridad & Baja \\
	\hline
	Precondición & Haber iniciado sesión. \\
	\hline
	Descripción & Dentro de la app se podrán configurar aspectos como el tema de la aplicación (claro/oscuro). \\
	\hline
	Entrada & Nuevos datos de configuración. \\
	\hline
	Salida & Configuración modificada. \\
	\hline
	Secuencia normal & \begin{tabular}{@{}p{1cm}|p{9.5cm}@{}}
		Paso & Acción \\
		\hline  
		1 & El actor se dirigirá a la sección de ‘Mi Perfil’ y entre las opciones seleccionará ‘Configuración de la aplicación’. \\
		\hline  
		2 & Dentro de esta pantalla el usuario cambiará los valores deseados. \\
		\end{tabular} \\
	\hline
	Postcondición & Se han cambiado los parámetros de la aplicación deseados. \\
	\hline
	Excepciones & \begin{tabular}{@{}p{1cm}|p{9.5cm}@{}}
		Paso & Acción \\
		\hline  
		2 & Los nuevos parámetros de configuración no son correctos. Se mantiene la configuración anterior. \\
		\end{tabular}  \\
	\hline
	Comentarios & Este caso de uso permite modificar la configuración de la aplicación. \\
	\hline
	Actores & Usuario, Profesional \\
	\hline            
	\end{tabularx}
	\caption{Configurar app}
	\label{tab:cu_4}  
\end{table}
%---------------------------------------------------------------
\newpage
\begin{table}[!h]
	\begin{tabularx}{\textwidth}{|c|X|}
	\rowcolor[HTML]{00D2CB} 
	\hline          
	\textbf{Requisito} & \textbf{Ver lista de favoritos} \\
	\hline
	Identificador & 1.5 \\
	\hline
	Prioridad & Media \\
	\hline
	Precondición & El Usuario o Profesional debe estar autenticado en la aplicación y tener al menos un elemento en su lista de favoritos. \\
	\hline
	Descripción & Permite al Usuario o Profesional ver la lista de elementos marcados como favoritos (por ejemplo, Clientes o Profesionales) para un acceso rápido y conveniente. \\
	\hline
	Entrada & Selección de la opción para ver la lista de favoritos. \\
	\hline
	Salida & Lista de elementos marcados como favoritos con detalles relevantes. \\
	\hline
	Secuencia normal & \begin{tabular}{@{}p{1cm}|p{9.5cm}@{}}
		Paso & Acción \\
		\hline  
		1 & El Usuario o Profesional accede a la sección de lista de favoritos. \\
		\hline  
		2 & Selecciona la opción para ver la lista de favoritos. \\
		\hline  
		3 & Visualiza la lista de elementos marcados como favoritos con sus detalles. \\
		\end{tabular} \\
	\hline
	Postcondición & El Usuario o Profesional ve la lista de elementos marcados como favoritos. \\
	\hline
	Excepciones & \begin{tabular}{@{}p{1cm}|p{9.5cm}@{}}
		Paso & Acción \\
		\hline  
		2 & Si la lista de favoritos está vacía, se muestra un mensaje indicando que no hay elementos en la lista. \\
		\end{tabular}  \\
	\hline
	Comentarios & N.A. \\
	\hline
	Actores & Usuario, profesional \\
	\hline            
	\end{tabularx}
	\caption{Ver lista de favoritos}
	\label{tab:cu_5}  
\end{table}
%---------------------------------------------------------------
\newpage
\begin{table}[!h]
	\begin{tabularx}{\textwidth}{|c|X|}
	\rowcolor[HTML]{00D2CB} 
	\hline          
	\textbf{Requisito} & \textbf{Valorar cliente/profesional} \\
	\hline
	Identificador & 1.6 \\
	\hline
	Prioridad & Alta \\
	\hline
	Precondición & Haber iniciado sesión, haber recibido el trabajo de un profesional o haber realizado un trabajo a un usuario. \\
	\hline
	Descripción & Una vez realizado/recibido un trabajo, los usuarios/profesionales podrán valorar a ese profesional/usuario así como denunciarlo si lo vieran necesario. \\
	\hline
	Entrada & Datos de valoración, mensaje con más detalles (opcional) \\
	\hline
	Salida & N.A. \\
	\hline
	Secuencia normal & \begin{tabular}{@{}p{1cm}|p{9.5cm}@{}}
		Paso & Acción \\
		\hline  
		1 & Una vez recibido/realizado el trabajo, se mostrará la opción de valorar al profesional/usuario. \\
		\hline  
		2 & El usuario/profesional rellena el formulario de valoración. \\
		\end{tabular} \\
	\hline
	Postcondición & Se ha valorado al profesional/usuario. \\
	\hline
	Excepciones & \begin{tabular}{@{}p{1cm}|p{9.5cm}@{}}
		Paso & Acción \\
		\hline  
		2 & Los datos de entrada no son correctos. No se completa la valoración. \\
		\end{tabular}  \\
	\hline
	Comentarios & N.A. \\
	\hline
	Actores & Usuario, profesional \\
	\hline            
	\end{tabularx}
	\caption{Valorar cliente/profesional}
	\label{tab:cu_6}  
\end{table}
%---------------------------------------------------------------
\newpage
\begin{table}[!h]
	\begin{tabularx}{\textwidth}{|c|X|}
	\rowcolor[HTML]{00D2CB} 
	\hline          
	\textbf{Requisito} & \textbf{Chatear con cliente/profesional} \\
	\hline
	Identificador & 1.7 \\
	\hline
	Prioridad & Media \\
	\hline
	Precondición & Haber iniciado sesión, estar en el perfil de un profesional/cliente. \\
	\hline
	Descripción & En el perfil de los profesionales/clientes habrá un botón para iniciar un chat privado que servirá para aclarar dudas, pedir/dar presupuestos para cosas concretas, etc…  \\
	\hline
	Entrada & botón de chat, mensaje(s) a enviar. \\
	\hline
	Salida & N.A. \\
	\hline
	Secuencia normal & \begin{tabular}{@{}p{1cm}|p{9.5cm}@{}}
		Paso & Acción \\
		\hline  
		1 & Una vez en el perfil del profesional/usuario se pulsa el botón ‘Chat privado’. \\
		\hline  
		2 & Al hacerlo, se abre un chat entre usuario y profesional donde se podrán enviar mensajes. \\
		\hline  
		3 & El usuario/profesional envía los mensajes deseados. \\
		\end{tabular} \\
	\hline
	Postcondición & Se ha creado un chat privado entre usuario y profesional. \\
	\hline
	Excepciones & N.A.\\
	\hline
	Comentarios & Este caso de uso permite establecer una conversación entre usuario y profesional. \\
	\hline
	Actores & Usuario, profesional \\
	\hline            
	\end{tabularx}
	\caption{Chatear con profesional}
	\label{tab:cu_7}  
\end{table}

\newpage
\subsection{Casos de uso de usuarios}

\begin{table}[!h]
	\begin{tabularx}{\textwidth}{|c|X|}
	\rowcolor[HTML]{00D2CB} 
	\hline          
	\textbf{Requisito} & \textbf{Configurar búsqueda} \\
	\hline
	Identificador & 2.1 \\
	\hline
	Prioridad & Media \\
	\hline
	Precondición & Haber iniciado sesión. \\
	\hline
	Descripción & A la hora de buscar el servicio de un profesional, los usuarios tendrán la opción de configurar los parámetros de la búsqueda como por ejemplo, seleccionar una categoría u ordenar por valoración. \\
	\hline
	Entrada & Datos de configuración de búsqueda. \\
	\hline
	Salida & N.A. \\
	\hline
	Secuencia normal & \begin{tabular}{@{}p{1cm}|p{9.5cm}@{}}
		Paso & Acción \\
		\hline  
		1 & Dentro de la aplicación, el usuario seleccionará la opción ‘Buscar profesional’. \\
		\hline  
		2 & Se configurarán todos los parámetros de la búsqueda y se pulsará el botón ‘Buscar’. \\
		\end{tabular} \\
	\hline
	Postcondición & Se han configurado los parámetros de la búsqueda. \\
	\hline
	Excepciones & N.A.  \\
	\hline
	Comentarios & Este caso de uso permite establecer unos parámetros determinados de búsqueda. \\
	\hline
	Actores & Usuario \\
	\hline            
	\end{tabularx}
	\caption{Configurar búsqueda}
	\label{tab:cu_8}  
\end{table}
%---------------------------------------------------------------
\begin{table}[!h]
	\begin{tabularx}{\textwidth}{|c|X|}
	\rowcolor[HTML]{00D2CB} 
	\hline          
	\textbf{Requisito} & \textbf{Buscar servicio} \\
	\hline
	Identificador & 2.2 \\
	\hline
	Prioridad & Media \\
	\hline
	Precondición & Haber iniciado sesión, haber configurado la búsqueda. \\
	\hline
	Descripción & Una vez configurados los parámetros de búsqueda, se mostrarán los servicios que cumplen los parámetros de la búsqueda.  \\
	\hline
	Entrada & Servicio a buscar. \\
	\hline
	Salida & Resultados de servicios que coinciden con la búsqueda. \\
	\hline
	Secuencia normal & \begin{tabular}{@{}p{1cm}|p{9.5cm}@{}}
		Paso & Acción \\
		\hline  
		1 & El usuario podrá recorrer una lista con todos los servicios ofrecidos que cumplen los filtros establecidos. \\
		\end{tabular} \\
	\hline
	Postcondición & Se ha realizado una búsqueda de servicio. \\
	\hline
	Excepciones & N.A.\\
	\hline
	Comentarios & Este caso de uso sirve para que los usuarios puedan buscar servicios de acuerdo con unos parámetros \\
	\hline
	Actores & Usuario \\
	\hline            
	\end{tabularx}
	\caption{Buscar servicio}
	\label{tab:cu_9}  
\end{table}
%---------------------------------------------------------------
\newpage
\begin{table}[!h]
	\begin{tabularx}{\textwidth}{|c|X|}
	\rowcolor[HTML]{00D2CB} 
	\hline          
	\textbf{Requisito} & \textbf{Consultar profesional} \\
	\hline
	Identificador & 2.3 \\
	\hline
	Prioridad & Media \\
	\hline
	Precondición & Haber iniciado sesión, hay profesionales disponibles. \\
	\hline
	Descripción & Los usuarios tendrán la posibilidad de consultar el perfil de los distintos profesionales: sus datos (categoría, honorarios, etc…), sus valoraciones y otras estadísticas.  \\
	\hline
	Entrada & Profesional a consultar. \\
	\hline
	Salida & Detalles del profesional consultado \\
	\hline
	Secuencia normal & \begin{tabular}{@{}p{1cm}|p{9.5cm}@{}}
		Paso & Acción \\
		\hline  
		1 & El usuario realiza una búsqueda de profesional o escoge uno de los que se muestra en el mapa. \\
		\hline  
		2 & Al pulsar en el profesional se abrirá su perfil con todos los datos de ese profesional. \\
		\end{tabular} \\
	\hline
	Postcondición & Se ha consultado el profesional seleccionado. \\
	\hline
	Excepciones & N.A.\\
	\hline
	Comentarios & Este caso de uso permite obtener más detalles acerca de un profesional determinado. \\
	\hline
	Actores & Usuario \\
	\hline            
	\end{tabularx}
	\caption{Chatear con profesional}
	\label{tab:cu_10}  
\end{table}
%---------------------------------------------------------------
\newpage
\begin{table}[!h]
	\begin{tabularx}{\textwidth}{|c|X|}
	\rowcolor[HTML]{00D2CB} 
	\hline          
	\textbf{Requisito} & \textbf{Contratar servicio} \\
	\hline
	Identificador & 2.4 \\
	\hline
	Prioridad & Alta \\
	\hline
	Precondición & Haber iniciado sesión. \\
	\hline
	Descripción & Cuando un usuario encuentre el servicio que necesita tendrá la posibilidad de contratarlo y el profesional decidirá si tomar o no el trabajo.  \\
	\hline
	Entrada & Servicio elegido, respuesta del profesional. \\
	\hline
	Salida & N.A. \\
	\hline
	Secuencia normal & \begin{tabular}{@{}p{1cm}|p{9.5cm}@{}}
		Paso & Acción \\
		\hline  
		1 & Una vez seleccionado el profesional, en la pantalla del perfil del mismo habrá una opción llamada ‘Contratar servicio’, el usuario pulsará el botón. \\
		\hline  
		2 & El sistema procesa y valida la contratación. \\
		\hline  
		3 & La propuesta de trabajo llega al profesional que decide si la acepta o rechaza. \\
		\end{tabular} \\
	\hline
	Postcondición & Se ha contratado el servicio. \\
	\hline
	Excepciones & \begin{tabular}{@{}p{1cm}|p{9.5cm}@{}}
		Paso & Acción \\
		\hline  
		2 & Fallo en el proceso y validación de la petición. Se avisa al usuario y no se envía la propuesta. \\
		\end{tabular}\\
	\hline
	Comentarios & N.A. \\
	\hline
	Actores & Usuario \\
	\hline            
	\end{tabularx}
	\caption{Contratar servicio}
	\label{tab:cu_11}  
\end{table}
%---------------------------------------------------------------
\newpage
\begin{table}[!h]
	\begin{tabularx}{\textwidth}{|c|X|}
	\rowcolor[HTML]{00D2CB} 
	\hline          
	\textbf{Requisito} & \textbf{Añadir/Quitar Profesional de lista de favoritos} \\
	\hline
	Identificador & 2.5 \\
	\hline
	Prioridad & Media \\
	\hline
	Precondición & Haber iniciado sesión, estar en el perfil de un profesional. \\
	\hline
	Descripción & Los usuarios tendrán la posibilidad de añadir o quitar a profesionales de su lista de favoritos, lista en la que podrán tener un acceso rápido a los perfiles de dichos profesionales.  \\
	\hline
	Entrada & Botón de añadir/quitar de favoritos. \\
	\hline
	Salida & Lista de favoritos modificada. \\
	\hline
	Secuencia normal & \begin{tabular}{@{}p{1cm}|p{9.5cm}@{}}
		Paso & Acción \\
		\hline  
		1 & Una vez dentro del perfil de un profesional, el usuario dispondrá de un botón de favoritos. \\
		\hline  
		2 & Dependiendo de si quiere añadir o quitar al profesional de favorito lo activará o desactivará. \\
		\end{tabular} \\
	\hline
	Postcondición & Se ha añadido/quitado un profesional en la lista de favoritos. \\
	\hline
	Excepciones & N.A.\\
	\hline
	Comentarios & N.A. \\
	\hline
	Actores & Usuario \\
	\hline            
	\end{tabularx}
	\caption{Contratar servicio}
	\label{tab:cu_12}  
\end{table}

\newpage
\subsection{Casos de uso de profesionales}
\begin{table}[!h]
	\begin{tabularx}{\textwidth}{|c|X|}
	\rowcolor[HTML]{00D2CB} 
	\hline          
	\textbf{Requisito} & \textbf{Cambiar estado de profesional} \\
	\hline
	Identificador & 3.1 \\
	\hline
	Prioridad & Alta \\
	\hline
	Precondición & Haber iniciado sesión como profesional. \\
	\hline
	Descripción & El profesional puede cambiar su estado entre activo/inactivo/trabajando para indicar a los usuarios su disponibilidad actual. \\
	\hline
	Entrada & Nuevo estado. \\
	\hline
	Salida & N.A. \\
	\hline
	Secuencia normal & \begin{tabular}{@{}p{1cm}|p{9.5cm}@{}}
		Paso & Acción \\
		\hline  
		1 & El profesional se dirige al apartado de ‘Mi Perfil’. \\
		\hline  
		2 & Dentro del mismo selecciona la opción ‘Cambiar estado’. \\
		\hline  
		3 & El profesional selecciona el nuevo estado que figura en su perfil. \\
		\end{tabular} \\
	\hline
	Postcondición & Se ha cambiado el estado de profesional. \\
	\hline
	Excepciones & N.A. \\
	\hline
	Comentarios & N.A. \\
	\hline
	Actores & Profesional \\
	\hline            
	\end{tabularx}
	\caption{Cambiar estado de profesional}
	\label{tab:cu_13}  
\end{table}
%---------------------------------------------------------------
\newpage
\begin{table}[!h]
	\begin{tabularx}{\textwidth}{|c|X|}
	\rowcolor[HTML]{00D2CB} 
	\hline          
	\textbf{Requisito} & \textbf{Dar de alta/baja servicio} \\
	\hline
	Identificador & 3.2 \\
	\hline
	Prioridad & Alta \\
	\hline
	Precondición & Haber iniciado sesión como profesional. \\
	\hline
	Descripción & Los profesionales tendrán la opción de dar de alta y baja los servicios, en el primer caso esto significa que sube un servicio y en el segundo que lo retira de la oferta de servicios. \\
	\hline
	Entrada & Servicio a dar de baja/alta. \\
	\hline
	Salida & Confirmación de la baja/alta del servicio. \\
	\hline
	Secuencia normal: alta de servicio & \begin{tabular}{@{}p{1cm}|p{6.5cm}@{}}
		Paso & Acción \\
		\hline  
		1 & Cuando un usuario contrata el servicio de un profesional, a este le llega una solicitud de servicio. \\
		\hline  
		2 & El profesional abre la oferta, donde verá los detalles del servicio. \\
		\hline  
		3 & Pulsa el botón ‘Aceptar servicio’. \\
		\end{tabular} \\
	\hline
	Secuencia normal: baja de servicio & \begin{tabular}{@{}p{1cm}|p{6.5cm}@{}}
		Paso & Acción \\
		\hline  
		1 & El profesional abre el servicio que había aceptado con anterioridad. \\
		\hline  
		2 & Entre los detalles del servicio se muestra la opción ‘Dar de baja servicio’ el profesional pulsa el botón. \\
		\hline  
		3 & Se muestra un mensaje de confirmación de baja de servicio. \\
		\end{tabular} \\
	\hline
	Postcondición & Se ha dado de alta/baja un servicio. \\
	\hline
	Excepciones & N.A.  \\
	\hline
	Comentarios & N.A. \\
	\hline
	Actores & Profesional   \\
	\hline            
	\end{tabularx}
	\caption{Dar de alta/baja servicio}
	\label{tab:cu_14}  
\end{table}
%---------------------------------------------------------------
\newpage
\begin{table}[!h]
	\begin{tabularx}{\textwidth}{|c|X|}
	\rowcolor[HTML]{00D2CB} 
	\hline          
	\textbf{Requisito} & \textbf{Modificar servicio} \\
	\hline
	Identificador & 3.3 \\
	\hline
	Prioridad & Alta \\
	\hline
	Precondición & Haber iniciado sesión como profesional, tener al menos un servicio dado de alta. \\
	\hline
	Descripción & Permite al Profesional modificar la información de un servicio que ofrece. \\
	\hline
	Entrada & Detalles actualizados del servicio. \\
	\hline
	Salida & Confirmación de la modificación del servicio. \\
	\hline
	Secuencia normal & \begin{tabular}{@{}p{1cm}|p{9.5cm}@{}}
		Paso & Acción \\
		\hline  
		1 & El profesional navega hasta la sección de gestión de servicios. \\
		\hline  
		2 & Selecciona el servicio que desea modificar. \\
		\hline  
		3 & Realiza las modificaciones necesarias en los detalles del servicio. \\
		\hline  
		4 & Confirma la acción de modificación. \\
		\end{tabular} \\
	\hline
	Postcondición & El servicio se actualiza con la nueva información en el perfil del Profesional. \\
	\hline
	Excepciones & \begin{tabular}{@{}p{1cm}|p{9.5cm}@{}}
		Paso & Acción \\
		\hline  
		3 & Si no se proporciona la información necesaria, se muestra un mensaje de error. \\
		\hline  
		4 & Si la operación falla por algún motivo, se notifica al Profesional. \\
		\end{tabular}  \\
	\hline
	Comentarios & Este caso de uso permite a los Profesionales mantener actualizada la información de sus servicios. \\
	\hline
	Actores & Profesional   \\
	\hline            
	\end{tabularx}
	\caption{Modificar servicio}
	\label{tab:cu_15}  
\end{table}
%---------------------------------------------------------------
\newpage
\begin{table}[!h]
	\begin{tabularx}{\textwidth}{|c|X|}
	\rowcolor[HTML]{00D2CB} 
	\hline          
	\textbf{Requisito} & \textbf{Listar servicios dados de alta} \\
	\hline
	Identificador & 3.4 \\
	\hline
	Prioridad & Baja \\
	\hline
	Precondición & Haber iniciado sesión como profesional, tener al menos un servicio dado de alta. \\
	\hline
	Descripción & Permite al Profesional ver una lista de todos los servicios que ha dado de alta. \\
	\hline
	Entrada & Selección de la opción para listar servicios. \\
	\hline
	Salida & Lista de servicios con detalles. \\
	\hline
	Secuencia normal & \begin{tabular}{@{}p{1cm}|p{9.5cm}@{}}
		Paso & Acción \\
		\hline  
		1 & El profesional navega hasta la sección de gestión de servicios. \\
		\hline  
		2 & Selecciona la opción para listar sus servicios. \\
		\end{tabular} \\
	\hline
	Postcondición & El Profesional puede ver una lista de los servicios que ha dado de alta. \\
	\hline
	Excepciones & \begin{tabular}{@{}p{1cm}|p{9.5cm}@{}}
		Paso & Acción \\
		\hline  
		2 & Si no tiene servicios dados de alta, se muestra un mensaje indicando que no tiene servicios registrados. \\
		\end{tabular}  \\
	\hline
	Comentarios & N.A. \\
	\hline
	Actores & Profesional   \\
	\hline            
	\end{tabularx}
	\caption{Listar servicios dados de alta}
	\label{tab:cu_16}  
\end{table}
%---------------------------------------------------------------
\newpage
\begin{table}[!h]
	\begin{tabularx}{\textwidth}{|c|X|}
	\rowcolor[HTML]{00D2CB} 
	\hline          
	\textbf{Requisito} & \textbf{Contestar solicitud de contratación.} \\
	\hline
	Identificador & 3.5 \\
	\hline
	Prioridad & Alta \\
	\hline
	Precondición & El Profesional debe estar autenticado en la aplicación y haber recibido una solicitud de contratación. \\
	\hline
	Descripción & Permite al Profesional aceptar o rechazar una solicitud de contratación de un Usuario. \\
	\hline
	Entrada & Solicitud de contratación. \\
	\hline
	Salida & Confirmación de la respuesta a la solicitud. \\
	\hline
	Secuencia normal & \begin{tabular}{@{}p{1cm}|p{9.5cm}@{}}
		Paso & Acción \\
		\hline  
		1 & El profesional recibe una notificación de solicitud de contratación. \\
		\hline  
		2 & Accede a la solicitud y selecciona aceptar o rechazar. \\
		\hline  
		3 & Confirma la respuesta. \\
		\end{tabular} \\
	\hline
	Postcondición & La solicitud de contratación se responde y se notifica al Usuario. \\
	\hline
	Excepciones & \begin{tabular}{@{}p{1cm}|p{9.5cm}@{}}
		Paso & Acción \\
		\hline  
		3 & Si la solicitud ha caducado, se informa al Profesional. \\
		\end{tabular}  \\
	\hline
	Comentarios & N.A. \\
	\hline
	Actores & Usuario, Profesional   \\
	\hline            
	\end{tabularx}
	\caption{Contestar solicitud de contratación.}
	\label{tab:cu_17}  
\end{table}
%---------------------------------------------------------------
\newpage
\begin{table}[!h]
	\begin{tabularx}{\textwidth}{|c|X|}
	\rowcolor[HTML]{00D2CB} 
	\hline          
	\textbf{Requisito} & \textbf{Consultar cliente} \\
	\hline
	Identificador & 3.6 \\
	\hline
	Prioridad & Media \\
	\hline
	Precondición & Haber iniciado sesión como profesional y estar trabajando o haber trabajado con el cliente. \\
	\hline
	Descripción & Permite al Profesional consultar información sobre el cliente, incluyendo sus datos, las valoraciones que ha recibido y otras estadísticas relevantes. \\
	\hline
	Entrada & Selección del Cliente a consultar. \\
	\hline
	Salida & Información detallada del Cliente, incluyendo sus datos personales, valoraciones recibidas y estadísticas. \\
	\hline
	Secuencia normal & \begin{tabular}{@{}p{1cm}|p{9.5cm}@{}}
		Paso & Acción \\
		\hline  
		1 & El profesional accede a la sección de consulta de Clientes. \\
		\hline  
		2 & Selecciona el Cliente cuya información desea consultar. \\
		\hline  
		3 & Visualiza los datos y estadísticas del cliente. \\
		\end{tabular} \\
	\hline
	Postcondición & El Profesional obtiene información sobre el cliente. \\
	\hline
	Excepciones & N.A.\\
	\hline
	Comentarios & Este caso de uso brinda al Profesional acceso a información relevante sobre los Clientes con los que ha interactuado. \\
	\hline
	Actores & Usuario, Profesional \\
	\hline            
	\end{tabularx}
	\caption{Consultar cliente.}
	\label{tab:cu_18}  
\end{table}
%---------------------------------------------------------------
\newpage
\begin{table}[!h]
	\begin{tabularx}{\textwidth}{|c|X|}
	\rowcolor[HTML]{00D2CB} 
	\hline          
	\textbf{Requisito} & \textbf{Añadir/Modificar/Quitar cliente de lista de favoritos.} \\
	\hline
	Identificador & 3.7 \\
	\hline
	Prioridad & Media \\
	\hline
	Precondición & Haber iniciado sesión como profesional. \\
	\hline
	Descripción & Permite al Profesional agregar, modificar o quitar Clientes de su lista de favoritos para un acceso más rápido y conveniente. \\
	\hline
	Entrada & Selección de la acción (añadir, modificar o quitar) y Cliente seleccionado. \\
	\hline
	Salida & N.A. \\
	\hline
	Secuencia normal & \begin{tabular}{@{}p{1cm}|p{9.5cm}@{}}
		Paso & Acción \\
		\hline  
		1 & El profesional navega hasta la sección de gestión de favoritos. \\
		\hline  
		2 & Selecciona la acción deseada (añadir, modificar o quitar) y el cliente correspondiente. \\
		\hline  
		3 & Confirma la acción. \\
		\end{tabular} \\
	\hline
	Postcondición & La lista de favoritos se actualiza según la acción realizada. \\
	\hline
	Excepciones & \begin{tabular}{@{}p{1cm}|p{9.5cm}@{}}
		Paso & Acción \\
		\hline  
		2 & Si el Cliente ya está en la lista de favoritos y se selecciona ‘añadir’, se muestra un mensaje informativo. \\
		\end{tabular}  \\
	\hline
	Comentarios & Este caso de uso permite al Profesional gestionar su lista de favoritos para un acceso más rápido a los Clientes preferidos. \\
	\hline
	Actores & Usuario, Profesional   \\
	\hline            
	\end{tabularx}
	\caption{Añadir/Modificar/Quitar cliente de lista de favoritos.}
	\label{tab:cu_19}  
\end{table}

\newpage
\subsection{Casos de uso de administrador}
\begin{table}[!h]
	\begin{tabularx}{\textwidth}{|c|X|}
	\rowcolor[HTML]{00D2CB} 
	\hline          
	\textbf{Requisito} & \textbf{Añadir/Eliminar/Modificar categorías de servicios ofrecidos.} \\
	\hline
	Identificador & 4.1 \\
	\hline
	Prioridad & Alta \\
	\hline
	Precondición & Estar registrado en la aplicación como administrador. \\
	\hline
	Descripción & Permite al Administrador gestionar las categorías de servicios ofrecidos, incluyendo la adición, eliminación o modificación de categorías existentes. \\
	\hline
	Entrada & Selección de la acción (añadir, eliminar o modificar) y detalles de la categoría. \\
	\hline
	Salida & Confirmación de la acción realizada en las categorías. \\
	\hline
	Secuencia normal & \begin{tabular}{@{}p{1cm}|p{9.5cm}@{}}
		Paso & Acción \\
		\hline  
		1 & El administrador accede a la sección de gestión de categorías de servicios. \\
		\hline  
		2 & Selecciona la acción deseada (añadir, eliminar o modificar) y proporciona los detalles necesarios. \\
		\hline  
		3 & Confirma la acción. \\
		\end{tabular} \\
	\hline
	Postcondición & Las categorías se actualizan según la acción realizada. \\
	\hline
	Excepciones & \begin{tabular}{@{}p{1cm}|p{9.5cm}@{}}
		Paso & Acción \\
		\hline  
		3 & Si la categoría ya existe y se selecciona ‘añadir’, se muestra un mensaje informativo. \\
		\hline  
		3 & Si la categoría no existe y se selecciona ‘eliminar’ o ‘modificar’, se muestra un mensaje informativo. \\
		\end{tabular} \\
	\hline
	Comentarios & Este caso de uso permite al Administrador gestionar las categorías de servicios para mantener la organización de la plataforma. \\
	\hline
	Actores & Administrador \\
	\hline            
	\end{tabularx}
	\caption{Añadir/eliminar/Modificar categorías de servicios ofrecidos.}
	\label{tab:cu_20}  
\end{table}
%---------------------------------------------------------------
\newpage
\begin{table}[!h]
	\begin{tabularx}{\textwidth}{|c|X|}
	\rowcolor[HTML]{00D2CB} 
	\hline          
	\textbf{Requisito} & \textbf{Consultar datos/estadísticas de profesionales/clientes.} \\
	\hline
	Identificador & 4.2 \\
	\hline
	Prioridad & Media \\
	\hline
	Precondición & Estar registrado en la aplicación como administrador. \\
	\hline
	Descripción & Permite al Administrador acceder a datos y estadísticas relacionadas con Profesionales y Clientes, lo que le permite realizar análisis y tomar decisiones informadas. \\
	\hline
	Entrada & Selección de Profesional o Cliente a consultar. \\
	\hline
	Salida & Información detallada y estadísticas del Profesional o Cliente seleccionado. \\
	\hline
	Secuencia normal & \begin{tabular}{@{}p{1cm}|p{9.5cm}@{}}
		Paso & Acción \\
		\hline  
		1 & El administrador accede a la sección de consulta de datos/estadísticas. \\
		\hline  
		2 & Selecciona el Profesional o Cliente cuya información desea consultar. \\
		\hline  
		3 & Visualiza los datos y estadísticas. \\
		\end{tabular} \\
	\hline
	Postcondición & El Administrador obtiene información detallada sobre el Profesional o Cliente seleccionado. \\
	\hline
	Excepciones & \begin{tabular}{@{}p{1cm}|p{9.5cm}@{}}
		Paso & Acción \\
		\hline  
		2 & Si no se encuentra información para el Profesional o Cliente seleccionado, se muestra un mensaje informativo. \\
		\end{tabular} \\
	\hline
	Comentarios & Este caso de uso proporciona al Administrador acceso a datos relevantes para la toma de decisiones y la gestión de la plataforma. \\
	\hline
	Actores & Administrador \\
	\hline            
	\end{tabularx}
	\caption{Consultar datos/estadísticas de profesionales/clientes.}
	\label{tab:cu_21}  
\end{table}
%---------------------------------------------------------------
\newpage
\begin{table}[!h]
	\begin{tabularx}{\textwidth}{|c|X|}
	\rowcolor[HTML]{00D2CB} 
	\hline          
	\textbf{Requisito} & \textbf{Buscar clientes/profesionales.} \\
	\hline
	Identificador & 4.3 \\
	\hline
	Prioridad & Media \\
	\hline
	Precondición & Estar registrado en la aplicación como administrador. \\
	\hline
	Descripción & Permite al Administrador buscar Clientes o Profesionales dentro de la aplicación según diversos criterios. \\
	\hline
	Entrada & Criterios de búsqueda. \\
	\hline
	Salida & Lista de Clientes o Profesionales que coinciden con los criterios de búsqueda. \\
	\hline
	Secuencia normal & \begin{tabular}{@{}p{1cm}|p{9.5cm}@{}}
		Paso & Acción \\
		\hline  
		1 & El administrador accede a la sección de búsqueda de Clientes o Profesionales. \\
		\hline  
		2 & Ingresa los criterios de búsqueda. \\
		\hline  
		3 & Realiza la búsqueda. \\
		\hline  
		4 & Visualiza la lista de resultados. \\
		\end{tabular} \\
	\hline
	Postcondición & El Administrador obtiene una lista de Clientes o Profesionales que coinciden con los criterios de búsqueda. \\
	\hline
	Excepciones & \begin{tabular}{@{}p{1cm}|p{9.5cm}@{}}
		Paso & Acción \\
		\hline  
		3 & Si no se encuentran resultados que coincidan con los criterios, se muestra un mensaje informativo. \\		
		\end{tabular} \\
	\hline
	Comentarios & Este caso de uso permite al Administrador buscar y acceder a perfiles de Clientes o Profesionales de manera eficiente. \\
	\hline
	Actores & Administrador \\
	\hline            
	\end{tabularx}
	\caption{Buscar clientes/profesionales.}
	\label{tab:cu_22}  
\end{table}
%---------------------------------------------------------------
\newpage
\begin{table}[!h]
	\begin{tabularx}{\textwidth}{|c|X|}
	\rowcolor[HTML]{00D2CB} 
	\hline          
	\textbf{Requisito} & \textbf{Modificar datos de clientes/profesionales/servicios ofrecidos.} \\
	\hline
	Identificador & 4.4 \\
	\hline
	Prioridad & Alta \\
	\hline
	Precondición & Estar registrado en la aplicación como administrador. \\
	\hline
	Descripción & Permite al administrador realizar modificaciones en los datos de clientes, profesionales o servicios ofrecidos cuando sea necesario. \\
	\hline
	Entrada & Selección del tipo de modificación (Cliente, Profesional o Servicio) y detalles de la modificación. \\
	\hline
	Salida & Confirmación de la modificación realizada. \\
	\hline
	Secuencia normal & \begin{tabular}{@{}p{1cm}|p{9.5cm}@{}}
		Paso & Acción \\
		\hline  
		1 & El administrador accede a la sección de modificación de datos. \\
		\hline  
		2 & Selecciona el tipo de modificación deseada (Cliente, Profesional o Servicio) y proporciona los detalles necesarios. \\
		\hline  
		3 & Confirma la modificación. \\
		\end{tabular} \\
	\hline
	Postcondición & Los datos se actualizan según la modificación realizada. \\
	\hline
	Excepciones & N.A.\\
	\hline
	Comentarios & Este caso de uso permite al Administrador gestionar y mantener actualizados los datos de la plataforma. \\
	\hline
	Actores & Administrador \\
	\hline            
	\end{tabularx}
	\caption{Modificar datos de clientes/profesionales/servicios ofrecidos.}
	\label{tab:cu_23}  
\end{table}
%---------------------------------------------------------------
\newpage
\begin{table}[!h]
	\begin{tabularx}{\textwidth}{|c|X|}
	\rowcolor[HTML]{00D2CB} 
	\hline          
	\textbf{Requisito} & \textbf{Dar de baja usuarios.} \\
	\hline
	Identificador & 4.5 \\
	\hline
	Prioridad & Alta \\
	\hline
	Precondición & Estar registrado en la aplicación como administrador. \\
	\hline
	Descripción & Permite al Administrador dar de baja a Usuarios de la aplicación en casos de incumplimiento de términos y condiciones u otras razones legítimas. \\
	\hline
	Entrada & Selección del Usuario a dar de baja y motivo de la baja. \\
	\hline
	Salida & Confirmación de la baja del Usuario. \\
	\hline
	Secuencia normal & \begin{tabular}{@{}p{1cm}|p{9.5cm}@{}}
		Paso & Acción \\
		\hline  
		1 & El administrador accede a la sección de gestión de bajas de Usuarios. \\
		\hline  
		2 & Selecciona el Usuario a dar de baja y especifica el motivo. \\
		\hline  
		3 & Confirma la baja del Usuario. \\
		\end{tabular} \\
	\hline
	Postcondición & El Usuario queda dado de baja de la aplicación. \\
	\hline
	Excepciones & N.A.\\
	\hline
	Comentarios & Este caso de uso permite al Administrador mantener la integridad de la plataforma al dar de baja a Usuarios que incumplen las reglas o políticas. \\
	\hline
	Actores & Administrador \\
	\hline            
	\end{tabularx}
	\caption{Dar de baja usuarios.}
	\label{tab:cu_24}  
\end{table}
