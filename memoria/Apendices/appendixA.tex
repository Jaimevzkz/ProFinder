\chapter{Especificación de requisitos}
\label{Appendix:srs}

En este apendice se especifican los actores y casos de uso de la aplicación Profinder. Se trata de una especificación inicial por lo que tanto actores como casos de uso han sido cambiados, ampliados o recortados a lo largo del proceso de desarrollo.

\section{Actores}
A continuación se describen los distintos actores de la aplicación:
\begin{itemize}
    \item \textbf{Usuario}: Un usuario es cualquier persona que se dé de alta como tal, debe haberse creado una cuenta e iniciado sesión para ser identificado, estos son los que contratan los servicios que los profesionales ofrecen, entre otras funciones los usuarios pueden ver el mapa de profesionales en tiempo real y filtrar los datos que le aparecen según la categoría de trabajo seleccionada.
    \item \textbf{Profesional}: Un profesional es cualquier persona que se dé de alta como tal, su proceso de registro es ligeramente diferente al de un usuario normal ya que deben especificar la categoría en la que son profesionales, así como otros detalles que sean importantes para su profesión.
    \item \textbf{Administrador}: El administrador de la aplicación es el que se encarga de la organización y el correcto funcionamiento de la misma, podrá añadir o eliminar categorías, bloquear usuarios y/o profesionales, así como responder a mensajes dirigidos a soporte, los administradores deben ser asignados una vez creada la cuenta desde fuera de la aplicación.
\end{itemize}
\section{Casos de Uso}
Los casos de uso se han clasificado en cuatro categorías según el actor principal del mismo.
\subsection{Índice de casos de uso}
\renewcommand{\labelenumii}{\theenumi.\arabic{enumii}.}
\begin{enumerate}
    \item Casos de uso generales
    \begin{enumerate}
        \item Registro/baja
        \item Modificar datos
        \item Login/logout
        \item Configurar app
        \item Ver lista de favoritos
        \item Valorar cliente/profesional
        \item Chatear con cliente/profesional
    \end{enumerate}
    \item Casos de uso de usuarios
    \begin{enumerate}
        \item Configurar búsqueda
        \item Buscar servicio
        \item Consultar profesional
        \item Contratar servicio
        \item Añadir/modificar/quitar profesional de lista de favoritos.
    \end{enumerate}
    \item Casos de uso de profesionales
    \begin{enumerate}
        \item Cambiar estado de profesional
        \item Dar de alta/baja servicio
        \item Modificar servicio
        \item Listar servicios dados de alta
        \item Contestar solicitud de contratación
        \item Consultar cliente
        \item Añadir/modificar/quitar cliente de lista de favoritos
    \end{enumerate}
    \item Casos de uso de administrador
    \begin{enumerate}
        \item Añadir/eliminar/modificar categorías de servicios ofrecidos.
        \item Consultar datos/estadísticas de profesional/clientes
        \item Buscar clientes/profesionales
        \item Modificar datos clientes/profesionales /servicios ofrecidos
        \item Dar de baja usuarios
    \end{enumerate}
\end{enumerate}
\newpage
\subsection{Casos de uso generales}
\begin{table}[!h]
	\begin{tabularx}{\textwidth}{|c|X|}
	\rowcolor[HTML]{00D2CB} 
	\hline          
	Requisito & Registro/baja \\
	\hline
	Identificador & 1.1 \\
	\hline
	Prioridad & Alta \\
	\hline
	Precondición & En caso de baja: tener una cuenta creada. \\
	\hline
	Descripción & Los actores de la aplicación tendrán que crear una cuenta para poder interactuar con la misma, así mismo tendrán la capacidad de dar de baja su cuenta cuando lo deseen. \\
	\hline
	Entrada & Nombre de usuario, correo electrónico, contraseña. \\
	\hline
	Salida & Actor registrado/dado de baja \\
	\hline
	Secuencia normal & \begin{tabular}{@{}p{2cm}|p{8.5cm}@{}}
		Paso & Acción \\
		\hline  
		1 & El actor abre la aplicación y se le muestra la pantalla para iniciar sesión con un botón específico para registrarse. \\
		\hline  
		2 & se selecciona la opción ‘Registrarse’. \\
		\hline  
		3 & El actor rellena el formulario con los datos de entrada y pulsa ‘Enviar’. \\
		\hline  
		4 & El sistema procesa el formulario y crea la cuenta. \\
		\end{tabular} \\
	\hline
	Postcondición & Se ha creado la cuenta. \\
	\hline
	Excepciones & \begin{tabular}{@{}p{2cm}|p{8.5cm}@{}}
		Paso & Acción \\
		\hline  
		4 & Los datos introducidos son incorrectos o no cumplen los requisitos. No se crea la cuenta y se avisa al usuario. \\
		\end{tabular}  \\
	\hline
	Comentarios & Este caso de uso permite registrar o dar de baja de la aplicación a actores de la misma. \\
	\hline
	Actores & Usuario, Profesional \\
	\hline            
	\end{tabularx}
	\caption{Registro/baja}
	\label{tab:cu_1}  
\end{table}

%---------------------------------------------------------------

\begin{table}[!h]
	\begin{tabularx}{\textwidth}{|c|X|}
	\rowcolor[HTML]{00D2CB} 
	\hline          
	Requisito & Modificar datos \\
	\hline
	Identificador & 1.2 \\
	\hline
	Prioridad & Media \\
	\hline
	Precondición & Haber iniciado sesión. \\
	\hline
	Descripción & Una vez creada una cuenta, tanto profesionales como usuarios podrán modificar sus datos de perfil. \\
	\hline
	Entrada & Datos a cambiar. \\
	\hline
	Salida & Nuevos datos. \\
	\hline
	Secuencia normal & \begin{tabular}{@{}p{2cm}|p{8.5cm}@{}}
		Paso & Acción \\
		\hline  
		1 & El actor se dirigirá al apartado de 'Mi perfil' y seleccionará la opción 'Modificar perfil'. \\
		\hline  
		2 & Dentro de esta pantalla se modificarán todos los datos deseados. \\
		\end{tabular} \\
	\hline
	Postcondición & Se han modificado los datos de perfil deseados. \\
	\hline
	Excepciones & \begin{tabular}{@{}p{2cm}|p{8.5cm}@{}}
		Paso & Acción \\
		\hline  
		2 & Los nuevos datos no son válidos. No se guarda la modificación. \\
		\end{tabular}  \\
	\hline
	Comentarios & Este caso de uso permite modificar datos de su perfil a los actores. \\
	\hline
	Actores & Usuario, Profesional \\
	\hline            
	\end{tabularx}
	\caption{Modificar datos}
	\label{tab:cu_2}  
\end{table}
\newpage
\subsection{Casos de uso de usuarios}

\newpage
\subsection{Casos de uso de profesionales}
\begin{table}[!h]
	\begin{tabularx}{\textwidth}{|c|X|}
	\rowcolor[HTML]{00D2CB} 
	\hline          
	Requisito & Modificar servicio \\
	\hline
	Identificador & 3.3 \\
	\hline
	Prioridad & Alta \\
	\hline
	Precondición & Haber iniciado sesión como profesional, tener al menos un servicio dado de alta. \\
	\hline
	Descripción & Permite al Profesional modificar la información de un servicio que ofrece. \\
	\hline
	Entrada & Detalles actualizados del servicio. \\
	\hline
	Salida & Confirmación de la modificación del servicio. \\
	\hline
	Secuencia normal & \begin{tabular}{@{}p{2cm}|p{8.5cm}@{}}
		Paso & Acción \\
		\hline  
		1 & El profesional navega hasta la sección de gestión de servicios. \\
		\hline  
		2 & Selecciona el servicio que desea modificar. \\
		\hline  
		3 & Realiza las modificaciones necesarias en los detalles del servicio. \\
		\hline  
		4 & Confirma la acción de modificación. \\
		\end{tabular} \\
	\hline
	Postcondición & El servicio se actualiza con la nueva información en el perfil del Profesional. \\
	\hline
	Excepciones & \begin{tabular}{@{}p{2cm}|p{8.5cm}@{}}
		Paso & Acción \\
		\hline  
		3 & Si no se proporciona la información necesaria, se muestra un mensaje de error. \\
		\hline  
		4 & Si la operación falla por algún motivo, se notifica al Profesional. \\
		\end{tabular}  \\
	\hline
	Comentarios & Este caso de uso permite a los Profesionales mantener actualizada la información de sus servicios. \\
	\hline
	Actores & Profesional   \\
	\hline            
	\end{tabularx}
	\caption{Modificar servicio}
	\label{tab:cu_15}  
\end{table}
\newpage
\subsection{Casos de uso de administrador}






