\chapter{Conclusiones y Trabajo Futuro}
\label{cap:conclusiones}
En este capítulo han sido plasmadas las conclusiones, así como las posibles líneas de trabajo futuro del proyecto.
\section{Conclusiones}
Desde que se planteó este trabajo de fin de grado en Julio de 2023, hasta la fecha en que se están redactando estas conclusiones han pasado aproximadamente 11 meses. Ha sido un viaje en el que se partía prácticamente de cero en cuanto a concimientos específicos de Android y en el que no se han parado de aprender conceptos nuevos mes a mes, casi se ha convertido en un hábito el dedicar todo el tiempo libre posible al proyecto y como consecuencia ha quedado una aplicación completa a pesar de tener muchas cosas que mejorar y diversas líneas de trabajo a seguir para hacerlo más completo, al fin y al cabo, este aprendizaje constante es de lo que trata el mundo de la programación y el desarrollo de software. Y es esto precisamente lo que lo hace tan apasionante.

A nivel del código desarrollado, a medida que se han ido añadiendo funcionalidades se han encontrado multitud de cosas a mejorar, y se ha hecho en la medida de lo posible teniendo en cuenta los ajustados tiempos, a esto ha ayudado una base estructural robusta que ha hecho posible que el proceso de refactorización haya sido siempre cómodo y eficiente. Este proyecto ha servido también para entender lo importante que es hacer las cosas bien, en los proyectos iniciales, cuando era necesario hacer un cambio, el proceso se convertía en un infierno, cada cambio hacía que otras cosas no funcionaran y acababa siendo más fácil deshechar el cambio. Por eso se tomó la determinación de que se evitaría por todos los medios que esto pasara con Profinder. La elección de las tecnologías (como se ha descrito en el capítulo \ref{cap:tecnologiasEmpleadas}) también ha sido de vital importancia, esto se ha notado por ejemplo con el uso de \hyperlink{subsec:kotlin}{Kotlin} y \hyperlink{subsec:compose}{Jeptack Compose}.

Profinder ha conseguido cumplir los objetivos propuestos inicialmente: Una aplicación capaz de actuar como intermediario entre distintos tipos de expertos profesionales que busquen ofrecer un servicio y clientes dispuestos a consumirlo. Ofreciendo también una serie de funcionalidades que hagan del proceso anterior algo fluido, descentralizado y sin comisiones.

\section{Líneas de trabajo futuro}
Como se ha mencionado anteriormente, Profinder es una aplicación completa, sin embargo, hay múltitud de puntos en los que se podría mejorar y/o completar. A continuación se han mencionado algunos:
\begin{itemize}
    \item \textbf{Notificaciones}: añadiría un toque extra de calidad y personalización a la aplicación el uso de notificaciones, especialmente para funcionalidades como el chat (notificaciones cuando se envían y reciben mensajes), la solicitud de servicios o los trabajos (cuando emipecen y terminen).
    \item \textbf{Animaciones}: el uso de las animaciones en Android es otra de las cosas que añade calidad a las aplicaciones, en Profinder no ha dado tiempo ha implementarlas en su totalidad ya que son un campo muy ámplio, sí que se han usado en algunos casos puntuales (como el efecto Shimmer que es una animación, aunque para ello se ha utilizado una biblioteca externa\footnote{Repositorio de la bilioteca Shimmer realizada por el usuario valentinilk: \href{https://github.com/valentinilk/compose-shimmer}{compose-shimmer}.}) pero la intención a futuro sería implementarlas en muchas otras partes de la aplicación.
    \item \textbf{Testing}: Como se mencionó en el apartado \ref{subsec:testing}, el testing de la aplicación ha sido el reto más difícil a superar y por falta de tiempo se ha quedado un poco a medias, la idea sería hacer un ámplio sistema de tests que incluya test unitarios, de integración y de UI para aportar una mayor robustez y conseguir eliminar los \textit{bugs} que puedan ir surgiendo.
    \item \textbf{Refactor  \hyperlink{subsec:mvi}{MVI}}: a pesar de que el patrón \hyperlink{subsec:mvi}{MVI} se ha aplicado de una forma correcta en la aplicación, de cara al final del proyecto se ha descubierto que se han cometido algunos errores (también denominados antipatrones) que podrían afectar a la escalabilidad en el futuro, debido a esto una refactorización que arregle estos problemas sería una de las líneas de trabajo futuro.
    \item \textbf{Playstore}: En un inicio, se planteó la posibilidad de publicar la aplicación en la Playstore pero al final no se llevó a cabo debido a que antes de hacerlo entra en juego la seguridad del código, habría que seguir procesos de ofuscamiento, control de entornos y CI/CD. No ha dado tiempo a estudiar con la profundidad necesaria estos temas por lo que se ha clasificado como línea de trabajo futuro.
    \item \textbf{Rol de administrador}: En la especificación de requisitos inicial se declaró el administrador como un actor que gestionara la aplicación desde dentro de la misma, a lo largo del proyecto no se consideró de suficiente importancia dentro de los plazos y se decidió hacer más incapié en otras partes de la aplicación. Sin embargo, podría ser interesante darle una vuelta a este concepto para establecer un mejor control de usuarios dentro de la aplicación.
    \item \textbf{Hacerla multiplataforma}: \href{https://kotlinlang.org/docs/multiplatform.html}{Kotlin multiplatform} es una nueva forma de hacer aplicaciones usando \hyperlink{subsec:kotlin}{Kotlin} y \hyperlink{subsec:compose}{Jeptack Compose}, a día de hoy no es una tecnología suficientemente madura como para ser viable pero permitiria hacer aplicaciones para distintos sistemas operativos (IOS, Windows...) con la misma base de código, esto permitiría aumentar considerablemente el alcance de la aplicación y la haría más completa. No se descarta migrarla a Kotlin multiplatform en el futuro.
\end{itemize}


