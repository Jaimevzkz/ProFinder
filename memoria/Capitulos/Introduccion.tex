\chapter{Introducción}
\label{cap:introduccion}

\chapterquote{We are stubborn on vision. We are flexible on details…}{Jeff Bezos}

\section{Motivación}
Profinder nace del concepto de agilizar y simplificar las relaciones entre profesionales que ofrecen un servicio y usuarios que están dispuestos a consumirlo.

Se trata de una aplicación móvil, inicialmente desarrollada 
para el sistema operativo android (pese a esto no se descarta la posibilidad de poder hacerla multiplataforma en un futuro, vease el capítulo \ref{cap:conclusiones}) en la que tanto usuarios como profesionales se podrán registrar, creando una cuenta e interactuar entre ellos. Las funcionalidades para cada tipo de actor varían en distintos aspectos, manteniendo algunas funcionalidades comunes.  

A continuación se explica la idea de flujo de publicación, contratación y calificación de servicios de la aplicación:
\begin{enumerate}
	\item Al registrarse, los profesionales seleccionaran la categoría de profesional a la que pertenecen -esta categoría podrá ser modificada en cualquier momento desde la pantalla de 'Editar perfil'-.
	\item Una vez registrados, tendrán la posibilidad de crear servicios que a su vez estarán clasificados en categorías y podrán ser públicos o privados (activos o inactivos).
	\item Los servicios activos aparecerán a los usuarios que podrán solicitarlos ya sea desde la pantalla de listado de servicios o a través del perfil del profesional.
	\item Al crear una solicitud un usuario, esta le aparece al profesional que podrá aceptarla o rechazarla. En caso de aceptarla, el trabajo se pone en marcha y en adelante hasta que termine se mostrará como trabajo activo.
	\item Una vez terminado el trabajo, el profesional lo marcará como tal y calificará con estrellas (del 1 al 5) al usuario. Para el profesional el trabajo habrá terminado.
	\item Por último al usuario le aparecerá la opción de calificar al profesional de la misma forma. Una vez hecho esto, el trabajo se marca como completado y se da por terminado el flujo de servicos.
\end{enumerate}
A parte de este flujo en el que cada actor tiene un papel marcado. Hay cierta funcionalidad común, descrita a continuación. Usuarios y profesionales podrán chatear entre sí usando la pantalla de chat, donde podrán concretar fechas y horarios concretos y detalles adicionales. Todos los usuarios y profesionales podrán ser añadidos a listas de favoritos para un fácil acceso a sus perfiles. El perfil de cada actor en la aplicación podrá ser completado con atributos como una descripción o una foto de perfil.

\section{Objetivos}
El mundo del desarrollo android es inmenso, la forma de diseñar interfaces respecto a la programación web muy distinta, y se trata de un mundo que en los últimos años ha estado en constante cambio, cada año salen nuevas tecnologías que dejan obsoleta la que ya había e incluso hace unos años cambió el lenguaje de programación usado para crear aplicaciones android.

El objetivo principal de este trabajo es aprender muchas de las cosas necesarias para ser un desarrollador android competente, partiendo de cero hasta llegar a ser capaz de crear una aplicación robusta, bonita y escalable en el tiempo. Los objetivos generales marcados desde el comienzo se muestran a continuación:
\begin{itemize}
	\item Aprender kotlin hasta alcanzar un nivel de competencia en el lenguaje apto para el desarrollo.
	\item Empezar en el desarrollo android utilizando el sistema de vistas \footnote{Artículo que hace una introducción al sistema de vistas \cite{androidViews} }, forma clásica de desarrollar interfaces en android, que sirve como primer acercamiento pese a que luego se sustituya por Jetpack Compose. 
\end{itemize}


\section{Plan de trabajo}
Aquí se describe el plan de trabajo a seguir para la consecución de los objetivos descritos en el apartado anterior.



\section{Explicaciones adicionales sobre el uso de esta plantilla}
Si quieres cambiar el \textbf{estilo del título} de los capítulos del documento, edita el fichero \verb|TeXiS\TeXiS_pream.tex| y comenta la línea \verb|\usepackage[Lenny]{fncychap}| para dejar el estilo básico de \LaTeX.

Si no te gusta que no haya \textbf{espacios entre párrafos} y quieres dejar un pequeño espacio en blanco, no metas saltos de línea (\verb|\\|) al final de los párrafos. En su lugar, busca el comando  \verb|\setlength{\parskip}{0.2ex}| en \verb|TeXiS\TeXiS_pream.tex| y aumenta el valor de $0.2ex$ a, por ejemplo, $1ex$.

TFGTeXiS se ha elaborado a partir de la plantilla de TeXiS\footnote{\url{http://gaia.fdi.ucm.es/research/texis/}}, creada por Marco Antonio y Pedro Pablo Gómez Martín para escribir su tesis doctoral. Para explicaciones más extensas y detalladas sobre cómo usar esta plantilla, recomendamos la lectura del documento \texttt{TeXiS-Manual-1.0.pdf} que acompaña a esta plantilla.

El siguiente texto se genera con el comando \verb|\lipsum[2-20]| que viene a continuación en el fichero .tex. El único propósito es mostrar el aspecto de las páginas usando esta plantilla. Quita este comando y, si quieres, comenta o elimina el paquete \textit{lipsum} al final de \verb|TeXiS\TeXiS_pream.tex|

\subsection{Texto de prueba}


\lipsum[2-20]