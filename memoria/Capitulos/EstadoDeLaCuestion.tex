\chapter{Estado de la Cuestión}
\label{cap:estadoDeLaCuestion}

\section{Uber}
\begin{figure}[h]
	\centering
	\includegraphics[width = 0.4\textwidth]{Imagenes/Fuentes/logo_Uber.png}
	\caption{Logotipo Uber}
	\label{fig:uber_logo}
\end{figure}
Uber \footnote{\url{https://www.uber.com}} es una aplicación muy conocida que sirve para contratar servicios de desplazamiento en V.T.C. (Vehículo de Transporte con Conductor), el motivo por el que se considera una aplicación relacionada con Profinder es que los servicios se contratan por proximidad, se utiliza la ubicación del usuario -al igual que se hace en Profinder- para encontrar los conductores más próximos, reduciendo así tiempos de espera y costes de desplazamiento innecesario. Uber se ha utilizado como una de las aplicaciones de referencia ya que representa muy bien el concepto que se ha buscado desde el principio con Profinder.

\section{Habitissimo}
\begin{figure}[h]
	\centering
	\includegraphics[width = 0.4\textwidth]{Imagenes/Fuentes/habitissimo_logo.jpg}
	\caption{Logotipo Habitissimo}
	\label{fig:habitissimo_logo}
\end{figure}
Habitissimo \footnote{\url{https://www.habitissimo.es/}} es la posible competencia más directa que se ha encontrado. Es una aplicación en la que distintos profesionales pueden publicar ofertas de servicios, que los usuarios pueden contratar pidiendo un presupuesto. Esta aplicación está más enfocada a profesiones artesanales (carpintería, pintura, albañilería...) a diferencia de Profinder, que no está enfocada a ningún sector particular, sino que cuenta con distintas categorías que pueden ir ampliándose y ajustándose con el tiempo. Profinder también busca diferenciarse por tener un proceso más ágil que no dependa de ningún tipo de tercero y en el que se pueda tener un contacto más directo entre usuarios y profesionales. 

\section{Upwork}
\begin{figure}[h]
	\centering
	\includegraphics[width = 0.4\textwidth]{Imagenes/Fuentes/logo_upwork.png}
	\caption{Logotipo Upwork}
	\label{fig:upwork_logo}
\end{figure}


\section{Booksy}
\begin{figure}[h]
	\centering
	\includegraphics[width = 0.4\textwidth]{Imagenes/Fuentes/logo_booksy.png}
	\caption{Logotipo Booksy}
	\label{fig:booksy_logo}
\end{figure}