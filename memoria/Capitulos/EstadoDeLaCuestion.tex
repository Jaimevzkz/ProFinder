\chapter{Estado de la Cuestión}
\label{cap:estadoDeLaCuestion}
En este capítulo, se ha realizado un analísis de otras aplicaciones en el mercado, viendo sus características generales y comprobando si se considerarían o no competencia de Profinder en el caso de que se tuviera la idea de comercializar en un futuro. 

En todos los casos analizados, se ha encontrado una diferencia significativa respecto a Profinder: todas son aplicaciones de código cerrado, mientras que Profinder, ha sido concebida desde sus inicios, como una aplicación de código abierto, con todo su código fuente y diseños abiertos al público general, para que cualquiera pueda ver y/o modificar el código (véase el apartado \ref{subsec:openSource}).
\section{Uber}
Uber\hyperlink{cap:biblio}{\endnote{\textbf{Uber}: \url{https://www.uber.com}}}, es una aplicación muy conocida que sirve para contratar servicios de desplazamiento en V.T.C. (Vehículo de Transporte con Conductor), el motivo por el que se considera una aplicación relacionada con Profinder es que los servicios se contratan por proximidad, se utiliza la ubicación del usuario -al igual que se hace en Profinder- para encontrar los conductores más próximos, reduciendo así, tiempos de espera y costes de desplazamiento innecesario. Uber se ha utilizado como una de las aplicaciones de referencia ya que representa muy bien, el concepto que se ha buscado desde el principio con Profinder.
\section{Habitissimo}
Habitissimo\hyperlink{cap:biblio}{\endnote{\textbf{Habitissimo}: \url{https://www.habitissimo.es}}},
es una aplicación que opera principalmente en España e Italia, en la que distintos profesionales pueden publicar ofertas de servicios, que los usuarios pueden contratar pidiendo un presupuesto. 

Es la posible competencia más directa que se ha encontrado. Está específicamente enfocada al sector de la reforma y reparación, a diferencia de Profinder, que no está enfocada a ningún sector particular, sino que cuenta con distintas categorías que pueden ir ampliándose y ajustándose con el tiempo. Profinder también busca diferenciarse por ser una aplicación global, al no depender de una infraestructura por países ya que no ofrece un servicio final sino una ayuda al contacto de usuarios y profesionales en su ámbito local. Asimismo, Profinder, tiene un proceso más ágil que no depende de ningún tipo de tercero y en el que se puede tener un contacto más directo entre usuarios y profesionales, disminuyendo así posibles comisiones.
\section{Booksy}
Booksy\hyperlink{cap:biblio}{\endnote{\textbf{Booksy}: \url{https://booksy.com/en-us}}}, es una aplicación centrada en el sector de la estética, en la que se pueden encontrar distintos profesionales como peluqueros, maquilladores o masajistas. Operan principalmente en Estados Unidos y cuentan con una batería específica de profesionales.

Las principales diferencias con Profinder es que están enfocados solo en belleza, ofreciendo un rango de categorías menor, a su vez, Booksy tiene una gestión centralizada de profesionales y servicios, no teniendo libertad los primeros para gestionar su trabajo de la forma en que deseen. Esto es también una desventaja repecto a Profinder, ya que al tener una gestión descentralizada, es una aplicación escalable a nivel internacional. 
\section{Upwork}

Upwork\hyperlink{cap:biblio}{\endnote{\textbf{Upwork}: \url{https://www.upwork.com}}}, es una plataforma que comenzó hace más de 20 años a operar, destinada a poner en contacto empresas con nuevos talentos, que busquen trabajo. Se situan como uno de los líderes en su sector, siendo este volumen alto de usuarios, una de sus principales ventajas frente a competidores.

El concepto de la aplicación, tiene algunas cosas en común con Profinder, en el sentido de crear un contacto directo entre contratante y contratado. Sin embargo, Upwork, se mueve en el contexto empresarial, dejando de lado el particular, por ello, no sería considerada competencia directa y pasaría más a un segundo plano en relación con este proyecto, sirviendo algunas de sus funcionalidades como referencia e inspiración.







