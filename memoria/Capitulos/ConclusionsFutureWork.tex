\chapter*{Conclusions and Future Work}
\label{cap:conclusions}
\addcontentsline{toc}{chapter}{Conclusions and Future Work}
In this chapter, conclusions have been drawn and possible future lines of work for the project have been outlined.

\section*{Conclusions}
Since this final degree project was proposed in July 2023, approximately 11 months have passed from these conclusions are being written. It has been a journey that started from having no specific knowledge of Android, where new concepts have been continuously learnt month by month. Dedicating all possible free time to the project almost became a habit, resulting in a well-rounded application despite having many aspects to improve and various lines of work to pursue to make it complete. Ultimately, this constant learning is what programming and software development are all about, and it is precisely this that makes it so exciting.

Regarding the developed code, as functionalities were added, many areas for improvement were found and addressed in the best way possible given the tight timelines. A robust structural base has made the refactoring process always comfortable and efficient. This project has also highlighted the importance of doing things right; in the initial projects, making a change often turned into a nightmare, with each change causing other things to break, making it easier to abandon the change. Therefore, the decision was made to avoid this at all costs with Profinder. The choice of technologies has also been crucial, noticeable, for example, with the use of \hyperlink{subsec:kotlin}{Kotlin} and \hyperlink{subsec:compose}{Jetpack Compose}.

Profinder has managed to meet the initially proposed objectives: an application capable of acting as an intermediary between different types of professional experts looking to offer a service and clients willing to consume it, offering functionalities that make the previous process smooth, decentralized, and commission-free.

\section*{Future Work Lines}
As mentioned earlier, Profinder is a complete application; however, there are many areas where it could be improved and/or completed. Below are some of them:
\begin{itemize}
    \item \textbf{Notifications}: Adding notifications would provide an extra touch of quality and personalization to the application, especially for functionalities such as the chat (notifications when messages are sent and received), service requests, or jobs (when they start and finish).
    \item \textbf{Animations}: The use of animations in Android is another aspect that adds quality to applications. In Profinder, there has not been enough time to implement them fully as it is a broad field. Animations have been used in some specific cases (like the Shimmer effect, which is an animation, although an external library was used for this\footnote{Shimmer library repository by user valentinilk: \href{https://github.com/valentinilk/compose-shimmer}{compose-shimmer}.}), but the intention for the future would be to implement them in many other parts of the application.
    \item \textbf{Testing}: As mentioned in section \ref{subsec:testing}, testing the application has been the most challenging task to overcome, and due to lack of time, it has remained somewhat incomplete. The idea would be to create a comprehensive testing system that includes unit tests, integration tests, and UI tests to provide greater robustness and eliminate potential bugs.
    \item \textbf{Refactor \hyperlink{subsec:mvi}{MVI}}: Although the \hyperlink{subsec:mvi}{MVI} pattern has been correctly applied in the application, towards the end of the project, some mistakes (also known as anti-patterns) were discovered that could affect scalability in the future. Therefore, a refactoring to address these issues would be one of the future work lines.
    \item \textbf{Playstore}: Initially, the possibility of publishing the app on the Playstore was considered, but it was ultimately not done because of security related issues. Code obfuscation, environment control, and CI/CD processes would need to be followed. There was not enough time to study these topics in the necessary depth, so it is classified as a future work line.
    \item \textbf{Administrator Role}: In the initial requirements specification, the administrator was declared as an actor who would manage the application from within. Throughout the project, it was not considered sufficiently important within the deadlines, and more emphasis was placed on other parts of the application. However, it could be interesting to revisit this concept to establish better user control within the application.
    \item \textbf{Making it Cross-Platform}: \href{https://kotlinlang.org/docs/multiplatform.html}{Kotlin Multiplatform} is a new way of making applications using \hyperlink{subsec:kotlin}{Kotlin} and \hyperlink{subsec:compose}{Jetpack Compose}. Today, it is not a sufficiently mature technology to be viable, but it would allow making applications for different operating systems (IOS, Windows, etc.) with the same codebase. This would significantly increase the application's reach and make it more complete. Migrating to Kotlin Multiplatform in the future is not ruled out.
\end{itemize}