%---------------------------------------------------------------------
%
%                      TeXiS_dedic.tex
%
%---------------------------------------------------------------------
%
% TeXiS_dedic.tex
% Copyright 2009 Marco Antonio Gomez-Martin, Pedro Pablo Gomez-Martin
%
% This file belongs to TeXiS, a LaTeX template for writting
% Thesis and other documents. The complete last TeXiS package can
% be obtained from http://gaia.fdi.ucm.es/projects/texis/
%
% This work may be distributed and/or modified under the
% conditions of the LaTeX Project Public License, either version 1.3
% of this license or (at your option) any later version.
% The latest version of this license is in
%   http://www.latex-project.org/lppl.txt
% and version 1.3 or later is part of all distributions of LaTeX
% version 2005/12/01 or later.
%
% This work has the LPPL maintenance status `maintained'.
% 
% The Current Maintainers of this work are Marco Antonio Gomez-Martin
% and Pedro Pablo Gomez-Martin
%
%---------------------------------------------------------------------
%
% Contiene la definición de las macros para crear las páginas de
% dedicatorias.
%
%---------------------------------------------------------------------

%%%
% Gestión de la configuración
%%%

% Primera dedicatoria
\newcommand{\dedicatoriaUno}[1]{
\def\dedicatoriaUnoVal{#1}
}

% Segunda dedicatoria
\newcommand{\dedicatoriaDos}[1]{
\def\dedicatoriaDosVal{#1}
}

%%%
% Configuración terminada
%%%


%%%
%% COMANDO PARA CREAR LAS DEDICATORIAS
%% CONTIENE TODO EL CÓDIGO LaTeX
%%%
\newcommand{\putDedicatoria}[1]{
% Si no hemos puesto aún el marcador en el PDF, lo ponemos
\ifx\marcadorDedicatorias\undefined
\ifpdf
   \pdfbookmark{Dedicatoria}{dedicatoria}
\fi
\def\marcadorDedicatorias{1}
\fi
% Primera pagina con la dedicatoria
\thispagestyle{empty}\mbox{}
\vspace*{4cm}
\begin{flushright}
#1
\end{flushright}
\newpage
% Segunda página, vacia
\thispagestyle{empty}\mbox{}
\newpage
} %\newcommand{\putDedicatoria}[1]

\newcommand{\makeDedicatorias}{
% Dedicatoria 1
\ifx\dedicatoriaUnoVal\undefined
\else
\putDedicatoria{\dedicatoriaUnoVal}
\fi
% Dedicatoria 2
\ifx\dedicatoriaDosVal\undefined
\else
\putDedicatoria{\dedicatoriaDosVal}
\fi
}



% \newcommand{\makeDedicatorias}{
% % Primera dedicatoria
% \ifx\dedicatoriaUnoVal\undefined
% \else
% \thispagestyle{empty}\mbox{}

% \vspace*{4cm}
% \begin{flushright}
% \dedicatoriaUnoVal
% \end{flushright}
% % \hfill \emph{Al duque de Béjar}

% % \hfill \emph{y\qquad\quad}

% % \hfill \emph{a tí, lector carísimo}

% \newpage
% \thispagestyle{empty}\mbox{}
% \fi % \ifx\dedicatoriaDosVal\undefined
% \newpage

% % Una segunda dedicatoria
% \ifx\dedicatoriaDosVal\undefined
% \else

% \thispagestyle{empty}\mbox{}

% \vspace*{4cm}
% \begin{flushright}
% \dedicatoriaDosVal
% \end{flushright}

% % \hfill \emph{I can't go to a restaurant and}

% % \hfill \emph{order food because I keep looking}

% % \hfill \emph{at the fonts on the menu.}

% % \hfill Donald Knuth

% \newpage
% \thispagestyle{empty}\mbox{}
% \newpage

% \fi % \ifx\dedicatoriaDosVal\undefined

% } % \newcommand{\makeDedicatorias}

%\newpage
% Variable local para emacs, para que encuentre el fichero
% maestro de compilación
%%%
%%% Local Variables:
%%% mode: latex
%%% TeX-master: "../Tesis.tex"
%%% End:
